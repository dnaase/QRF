\documentclass[12pt]{article}
\usepackage{geometry} % see geometry.pdf on how to lay out the page. There's lots.
\usepackage{hyperref} 
\usepackage{graphicx}
\usepackage{amssymb}
\usepackage{epstopdf}
\usepackage{enumitem}
\usepackage{color}
\linespread{1.5}

\usepackage[utf8]{inputenc}
\usepackage[english]{babel}
\usepackage[english]{isodate}
\usepackage{moreverb}
\usepackage[parfill]{parskip}
\renewcommand{\sfdefault}{ptm}
\pagenumbering{Roman}
\newpage

\pagenumbering{arabic}

\setdescription{labelsep=\textwidth}
\DeclareGraphicsRule{.tif}{png}{.png}{`convert #1 `dirname #1`/`basename #1 .tif`.png}
\geometry{a3paper} % or letter or a5paper or ... etc
% \geometry{landscape} % rotated page geometry

% See the ``Article customise'' template for come common customisations

{\Huge  \title{ \textbf{QRF User Guide }\\ For QRF v0.01}}
\author{Yaping Liu}
\date{Jul. 15. 2015} % delete this line to display the current date


%%quick start 
%%make QRF=""
%%cd $QRF
%% perl QRF_pipeline.pl test configure.txt


%%% BEGIN DOCUMENT
\begin{document}

\maketitle
\tableofcontents



{\color{red}{\section{Introduction}}}
QRF: a random forest model to boost meQTL/eQTL prediction power by using T statistics calculated from common eQTL mapping program (e.g. Matrix EQTL), probes genetic distance-expected genetic distance (recombination rate) and normalized HiC signal. QRF is a public available free software (MIT license) mainly written in Java and wrapped up by perl script. Copyright belongs to Computational Biology Group, CSAIL, MIT .

{\color{red}{\section{Prerequisites}}}
\begin{enumerate}
\item System: Already tested in Linux (CentOS 6) and Mac OSX 10 (10G memory is the minimum requirement). 
\item Java: Java(TM) SE Runtime Environment 1.6 (Linux, Mac OSX) or later.
\item Perl: perl scripts in the utilities require Perl v 5.8.8 or later.
\item Specify \$PATH: Specify QRF's root directory in the environment

\end{enumerate}

{\color{red}\section{Quick Start}}

{\color{blue}\subsection{Download QRF program.}}
Download zipped file from \url{http://compbio.mit.edu/QRF/}

{\color{blue}\subsection{Download additional input files for test}}
All of the input example file could be downloaded from our website: \url{http://compbio.mit.edu/QRF/}.  All *.zip files need to be unzipped firstly. configure.txt will be under software root directory. All the other files will be under ./test\_data directory.
\begin{enumerate}

\item configure.txt file: specify all of the input files name below and working directory 
\item SNP probes location file: SNP.locs. 
\item SNP information file: SNP.tab. 
\item CpG/Gene\_expression probes location file: DNAm.locs. 
\item CpG/Gene\_expression  information file: DNAm.tab. 
\item Covariant information file: cov.tab. 
\item Recombination rate big wig file: genetic\_map\_1kGv3\_test.hg19.bw. 
\item Recombination rate expectation index file: All\_length\_1M.all\_chr.hg19.permutate100.1kGv3.summary.txt (\emph{\color{red} Download separately}). 
\item KR normalized HiC signal file: GM12878\_chr1\_1kb.KRnorm\_raw\_test.sparseBed.txt. 
\item training file: training\_2k.txt 

\end{enumerate}

{\color{blue}\subsection{Run QRF program in terminal.}}

\begin{enumerate}
\item All of the input file should be put into the same directory specified in configure.txt. The following command would output meQTL calling result from matrixEQTL and result from QRF:
\begin{verbatimtab}
perl QRF_pipeline.pl test configure.txt
\end{verbatimtab}
\end{enumerate}


{\color{red}\section{Interpret output files}}

{\color{blue}\subsection{*.afterRandomForest.annotate.txt file }}
Each row is one pair of meQTL. Here is the explanation of each column:
\begin{enumerate}
\item[chr:] chromosome name
\item[start:] genomic coordinate.(0-based)
\item[end:] genomic coordinate.(1-based)
\item[rsid:] SNP rsid.
\item[cpg id:] CpG's id as in your input DNAm.locs.
\item[matrixEQTL:] FDR corrected p value by matrixEQTL
\item[QRF:] permutation p value by QRF

\end{enumerate}


{\color{red}\section{Input files format}}

\begin{description}

\item[configure.txt file] specify all of the input files name information and working directory. Modify it as you need.
\item[MatrixQtlResult] Optional, when you provided the matrixEQTL result file, QRF will not run matrixEQTL anymore. 
\item[SNP\_loc] SNP probes location file. As input in matrixEQTL. 
\item[gene\_loc] CpG/Gene\_expression probes location file. As input in matrixEQTL.  
\item[SNP\_tab] SNP  information file. As input in matrixEQTL.
\item[gene\_tab] CpG/Gene\_expression  information file. As input in matrixEQTL.
\item[covar\_tab] Covariant information file. As input in matrixEQTL.
\item[recombination\_bw] Recombination rate big wig file. Could be downloaded from 1000 genome project, Hapmap, UCSC table browser or our website.
\item[recombination\_expect] Recombination rate expectation index file. Could be download from our website. 
\item[HiC\_KRnorm] KR normalized HiC signal file. Could be converted by  "normalize\_hic2014\_to\_sparseBed.pl" script under perl directory (detailed described in Usage Detail section) 
\item[training\_file] training file for QRF, could be generated by mode 2 in QRF\_pipeline.pl or downloaded directly from QRF website

\end{description}

{\color{red}\section{Usage Detail}}
{\color{blue}\subsection{QRF\_pipeline.pl}}
 \begin{verbatimtab}
perl QRF_pipeline.pl [option] prefix configure.txt
\end{verbatimtab}


{\color{blue}\subsubsection{General options}}
\begin{description}
\item[-{}-help] Generate this help message.

\item[-{}-mode \begin{math}<\end{math} mode\_number \begin{math}>\end{math}]  1. Detect meQTL/eQTL; 2. Generate training file; (Default: {\tt `--mode 1'})

\item[-{}-region \begin{math}<\end{math} interval \begin{math}>\end{math}]  Specify the region for mode 1 or 2. Should be format like: chr1:1-1000 or chr1 (Default: {\tt `--region chr1'})

\item[-{}-fdr \begin{math}<\end{math} FDR \begin{math}>\end{math}]  Specify the false discovery rate for Matrix EQTL and QRF (Default: {\tt `--fdr 0.01'})

\item[-{}-mem \begin{math}<\end{math} memory \begin{math}>\end{math}]  Specify the number of gigabytes in the memory to use (Default: {\tt `--mem 15'})

\item[-{}-qrf\_path \begin{math}<\end{math} QRF\_path \begin{math}>\end{math}]  Specify the QRF root direcotry (Default: not specified. use environment variable \$QRF)


\end{description}

{\color{blue}\subsubsection{Options for mode 1, run QRF}}
\begin{description}

\item[-{}-tree\_num \begin{math}<\end{math} tree\_number \begin{math}>\end{math}]  Specify the number of tree used for QRF (Default: {\tt `--tree\_num 1000'})
\item[-{}-class\_index \begin{math}<\end{math} class\_index \begin{math}>\end{math}]  Specify the column number that are the label column for QRF (Default: {\tt `--class\_index 5'})
\item[-{}-label\_class \begin{math}<\end{math} label\_class \begin{math}>\end{math}]  Specify the class label name used for QRF (Default: {\tt `--label\_class meqtl'})
\item[-{}-permutation\_times \begin{math}<\end{math} permutation\_times \begin{math}>\end{math}]  Specify the number of permutation used for QRF. 0 or negative value means not enabled (Default: {\tt `--permutation\_times 1'})
\item[-{}-sep \begin{math}<\end{math} String \begin{math}>\end{math}]  Specify the string used to seperate the column (Default: tab delimit)
\item[-{}-sub\_sampling \begin{math}<\end{math} num\_of\_samples \begin{math}>\end{math}]  Specify the number of sample used for QRF (Default: not enabled, and use all of the samples)
\item[-{}-hic\_resolution \begin{math}<\end{math} hic\_resolution \begin{math}>\end{math}]  Specify the resolution of HiC signal used, default is 1kb (Default: {\tt `--hic\_resolution 1'})

\end{description}


{\color{blue}\subsubsection{Options for mode 2, generate training file}}
\begin{description}
\item[-{}-positive\_probes\_sampling \begin{math}<\end{math} number\_of\_positive\_probes \begin{math}>\end{math}]  Specify the number of positive probes used for QRF training (Default: {\tt `--positive\_probes\_sampling 10000'})
\item[-{}-negative\_probes\_sampling \begin{math}<\end{math} number\_of\_negative\_probes \begin{math}>\end{math}]  Specify the number of negative probes used for QRF training (Default: {\tt `--negative\_probes\_sampling 10000'})

\end{description}

{\color{blue}\subsection{normalize\_hic2014\_to\_sparseBed.pl}}
 \begin{verbatimtab}
perl normalize_hic2014_to_sparseBed.pl  chr1_1kb.KRnorm chr1_1kb.RAWobserved > chr1_1kb.KRnorm.sparseBed.txt
\end{verbatimtab}

{\color{blue}\subsubsection{Input}}
\begin{description}
\item[chr1\_1kb.KRnorm] KRnorm normalized efficient as provided in Rao et al. Cell 2014. (GSE63525)
\item[chr1\_1kb.RAWobserved] Raw count as provided in Rao et al. Cell 2014. (GSE63525)
\item[chr1\_1kb.KRnorm.sparseBed.txt] output HiC signal for QRF

\end{description}






{\color{red}\section{Build on source code}}
Source code is available on Github website:\\
https://github.com/dnaase/QRF\\
All of the required libraries are available in \url{https://github.com/dnaase/QRF/lib/}.

{\color{red}\section{Contact for help}}
For any of question on QRF, please join our google group \url{https://groups.google.com/d/forum/qrf-help/} for help

{\color{red}\section{Cite QRF}}
Please use the following publication to cite QRF:


\end{document}